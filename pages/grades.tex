
\begin{itemize}
    \item 17+ баллов (5) — отличный уровень владения математикой, как базовой, так и олимпиадной. С большой долей вероятности ребенок покажет очень хорошие результаты на экзамене и олимпиадах.
    \item 15--16 баллов (5-) — (при условии решения олимпиадной задачи) — уровень высокий, присутствуют все базовые навыки, что позволяет рассчитывать на высокие результаты на экзамене и олимпиадах.
    \item 14--16 баллов (5-) — (без олимпиадной задачи) — уровень знаний высокий, присутствуют все базовые навыки, что позволяет рассчитывать на хорошие результаты на экзамене, но на олимпиадах скорее всего нет.
    \item 13--14 баллов (4+) — (при условии олимпиадной задачи) — неплохой уровень знаний, хорошо работает голова, при правильном подходе и усилиях — возможны неплохие результаты, как на экзамене, так и на олимпиадах.
    \item 10--13 баллов (4) — уровень знаний чуть выше среднего, присутствуют локальные пробелы, которые можно доработать без серьёзных усилий. Можно рассчитывать на неплохие результаты на экзамене.
    \item 7--9 баллов (3) — невысокий уровень знаний, есть достаточное количество пробелов, но ситуация не критическая. При стабильных дополнительных занятиях возможно достижение средних или даже неплохих результатов.
    \item 4--6 баллов (2) — поверхностный уровень знаний, пробелы практически во всех важных темах. Нет возможности рассчитывать на высокие результаты по профильной математике в дальнейшем.
    \item 0--3 баллов (2) — уровень знаний околонулевой. Не представляется возможным рассчитывать даже на хорошую сдачу базового экзамена на 4--5.
\end{itemize}